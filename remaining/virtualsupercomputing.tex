\begin{conf-abstract}[]
{Virtual Super-Computing Over Internet}
{ 
Amit Batajoo
Nabaraj Adhikari
Naresh Adhikari
Prem Regmi
Rajan Bastola
}
{2007}
\indexauthors{ 
Amit Batajoo,
Nabaraj Adhikari,
Naresh Adhikari,
Prem Regmi,
Rajan Bastola
}
``Virtual Super-Computing Over Internet " is a model for distributing the chunks of a large problem that would require the computation power comparable to the supercomputers to the voluntarily donated processing power of numbers of computers connected through the internet and have them solve the problems independently and send the result to the master computer.

This model of computing consists of a Job Dispatcher in master computer. The Job Dispatcher assigns a separate job tasks to the different computer that comes to receive job. The Job Requester is called Slave. Each of these slaves receives the job to execute and transfer the result to the master computer. Finally, the master computer combines the result obtained from different slaves. It results the possibility of success in building the virtual super computer using normal computer over internet. Virtual Super Computing Over Internet is based on the high level Java sockets, Java RMI and the http protocol. For implementation we have chosen a problem of finding the prime numbers from some lower limit to some of the higher orders of 10.

The major issues that should be handled are partitioning of the problem, authentication and authorization, dispatching the job, synchronization, error handling, load balancing and communication between nodes.
\end{conf-abstract}