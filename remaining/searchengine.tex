 \begin{conf-abstract}[]
{Search Engine}
{
Aashish Prakash Shrestha
Anju Thapa
Bidur Devkota
Meena Shrestha
}
{2006}
\indexauthors{
Aashish Prakash Shrestha,
Anju Thapa,
Bidur Devkota,
Meena Shrestha
}
The report describes the development effort that has been put into the development of a search engine. Searching for something in a huge information repository like the Internet can be compared to trying to drink water from a firehose. Such a wealth of information is available that most people are unable to handle it and they get lost in the vast ocean of information. So, we need a system that guides people to finding the information that they desire. A search engine is a program designed to assist in finding information stored on a computer system such as the World Wide Web, inside a corporate or proprietary network or a personal computer. The search engine allows one to ask for content meeting specific criteria (typically those containing a given word or phase) and retrieves a list of references that match those criteria. Research was conducted to understand how a typical search engine operates; its underlying technologies. The working of major commercial search engines like Google and Yahoo was investigated. During our research, we came across many instances of search engines that were developed using reusable components like an off-the-shelf spider or an off-the-shelf indexing technology. The use of off-the-shelf reusable components was avoided because this being an academic project, effort was made to gain as much hands-on software developement knowledge as possible. The development endeavor is not as complete and efficient as commericial search engines but to engender an extensible set of artifacts and module to expand as a high quality commercial search engine.
  \end{conf-abstract}